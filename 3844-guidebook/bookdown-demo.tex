\documentclass[]{book}
\usepackage{lmodern}
\usepackage{amssymb,amsmath}
\usepackage{ifxetex,ifluatex}
\usepackage{fixltx2e} % provides \textsubscript
\ifnum 0\ifxetex 1\fi\ifluatex 1\fi=0 % if pdftex
  \usepackage[T1]{fontenc}
  \usepackage[utf8]{inputenc}
\else % if luatex or xelatex
  \ifxetex
    \usepackage{mathspec}
  \else
    \usepackage{fontspec}
  \fi
  \defaultfontfeatures{Ligatures=TeX,Scale=MatchLowercase}
\fi
% use upquote if available, for straight quotes in verbatim environments
\IfFileExists{upquote.sty}{\usepackage{upquote}}{}
% use microtype if available
\IfFileExists{microtype.sty}{%
\usepackage{microtype}
\UseMicrotypeSet[protrusion]{basicmath} % disable protrusion for tt fonts
}{}
\usepackage[margin=1in]{geometry}
\usepackage{hyperref}
\hypersetup{unicode=true,
            pdftitle={English 3844: Writing and Digital Media Guidebook},
            pdfauthor={Andy Lautenschlager},
            pdfborder={0 0 0},
            breaklinks=true}
\urlstyle{same}  % don't use monospace font for urls
\usepackage{natbib}
\bibliographystyle{apalike}
\usepackage{longtable,booktabs}
\usepackage{graphicx,grffile}
\makeatletter
\def\maxwidth{\ifdim\Gin@nat@width>\linewidth\linewidth\else\Gin@nat@width\fi}
\def\maxheight{\ifdim\Gin@nat@height>\textheight\textheight\else\Gin@nat@height\fi}
\makeatother
% Scale images if necessary, so that they will not overflow the page
% margins by default, and it is still possible to overwrite the defaults
% using explicit options in \includegraphics[width, height, ...]{}
\setkeys{Gin}{width=\maxwidth,height=\maxheight,keepaspectratio}
\IfFileExists{parskip.sty}{%
\usepackage{parskip}
}{% else
\setlength{\parindent}{0pt}
\setlength{\parskip}{6pt plus 2pt minus 1pt}
}
\setlength{\emergencystretch}{3em}  % prevent overfull lines
\providecommand{\tightlist}{%
  \setlength{\itemsep}{0pt}\setlength{\parskip}{0pt}}
\setcounter{secnumdepth}{5}
% Redefines (sub)paragraphs to behave more like sections
\ifx\paragraph\undefined\else
\let\oldparagraph\paragraph
\renewcommand{\paragraph}[1]{\oldparagraph{#1}\mbox{}}
\fi
\ifx\subparagraph\undefined\else
\let\oldsubparagraph\subparagraph
\renewcommand{\subparagraph}[1]{\oldsubparagraph{#1}\mbox{}}
\fi

%%% Use protect on footnotes to avoid problems with footnotes in titles
\let\rmarkdownfootnote\footnote%
\def\footnote{\protect\rmarkdownfootnote}

%%% Change title format to be more compact
\usepackage{titling}

% Create subtitle command for use in maketitle
\newcommand{\subtitle}[1]{
  \posttitle{
    \begin{center}\large#1\end{center}
    }
}

\setlength{\droptitle}{-2em}
  \title{English 3844: Writing and Digital Media Guidebook}
  \pretitle{\vspace{\droptitle}\centering\huge}
  \posttitle{\par}
  \author{Andy Lautenschlager}
  \preauthor{\centering\large\emph}
  \postauthor{\par}
  \predate{\centering\large\emph}
  \postdate{\par}
  \date{2018-08-28}

\usepackage{booktabs}
\usepackage{amsthm}
\makeatletter
\def\thm@space@setup{%
  \thm@preskip=8pt plus 2pt minus 4pt
  \thm@postskip=\thm@preskip
}
\makeatother

\usepackage{amsthm}
\newtheorem{theorem}{Theorem}[chapter]
\newtheorem{lemma}{Lemma}[chapter]
\theoremstyle{definition}
\newtheorem{definition}{Definition}[chapter]
\newtheorem{corollary}{Corollary}[chapter]
\newtheorem{proposition}{Proposition}[chapter]
\theoremstyle{definition}
\newtheorem{example}{Example}[chapter]
\theoremstyle{definition}
\newtheorem{exercise}{Exercise}[chapter]
\theoremstyle{remark}
\newtheorem*{remark}{Remark}
\newtheorem*{solution}{Solution}
\begin{document}
\maketitle

{
\setcounter{tocdepth}{1}
\tableofcontents
}
\hypertarget{introduction}{%
\chapter{Introduction}\label{introduction}}

Welcome to English 3844: Writing and Digital Media! In this class we
write blogs, create podcasts and videos, and then curate all of this
content on our own websites.

This booklet contains instructions and resources related to English
3844: Writing and Digital Media. Inside you'll find instructions on how
to install and use Atom text editor, GitHub Desktop, and GitHub pages,
as well as a few readings and a collection of audio and video
development resources.

I'll add more and more content to this booklet as the semester
progresses.

\hypertarget{readings}{%
\chapter{Readings}\label{readings}}

We won't have many readings this semester, but I have compiled a few
excerpts from longer works below.

\hypertarget{from-the-rhetorical-situation-by-lloyd-bitzer-1968}{%
\section{From ``The Rhetorical Situation'' by Lloyd Bitzer
(1968)}\label{from-the-rhetorical-situation-by-lloyd-bitzer-1968}}

\emph{The study of rhetoric dates back thousands of years, predating
even Socrates. Since then, countless scholars have tried to answer the
question, "What makes discourse} rhetorical\emph{?" Lloyd Bitzer offered
one of the clearest answers to that question in 1968. Below is a
collection of excerpts from his essay
\href{http://www.arts.uwaterloo.ca/~raha/309CWeb/Bitzer(1968).pdf}{``The
Rhetorical Situation''}.}

\textbf{Rhetoric alters reality}\\
In order to clarify rhetoric-as-essentially-related-to-situation, we
should acknowledge a viewpoint that is commonplace but fundamental: a
work of rhetoric is pragmatic; it comes into existence for the sake of
something beyond itself; it functions ultimately to produce action or
change in the world; it performs some task. In short, rhetoric is a mode
of altering reality, not by the direct application of energy to objects,
but by the creation of discourse which changes reality through the
mediation of thought and action. The rhetor alters reality by bringing
into existence a discourse of such a character that the audience, in
thought and action, is so engaged that it becomes mediator of change. In
this sense rhetoric is always persuasive.

\textbf{The rhetorical situation}\\
Let us now amplify the nature of situation by providing a formal
definition and examining constituents. Rhetorical situation may be
defined as a complex of persons, events, objects, and relations
presenting an actual or potential exigence which can be completely or
partially removed if discourse, introduced into the situation, can so
constrain human decision or action as to bring about the significant
modification of the exigence. Prior to the creation and presentation of
discourse, there are three constituents of any rhetorical situation: the
first is the exigence; the second and third are elements of the complex,
namely the audience to be constrained in decision and action, and the
constraints which influence the rhetor and can be brought to bear upon
the audience. Any exigence is an imperfection marked by urgency; it is a
defect, an obstacle, something waiting to be done, a thing which is
other than it should be.

\textbf{Exigence}\\
In any rhetorical situation there will be at least one controlling
exigence which functions as the organizing principle: it specifies the
audience to be addressed and the change to be effected. The exigence may
or may not be perceived clearly by the rhetor or other persons in the
situation; it may be strong or weak depending upon the clarity of their
perception and the degree of their interest in it; it may be real or
unreal depending on the facts of the case; it may be important or
trivial; it may be such that discourse can completely remove it, or it
may persist in spite of repeated modifications; it may be completely
familiar - one of a type of exigences occurring frequently in our
experience - or it may be totally new, unique. When it is perceived and
when it is strong and important, then it constrains the thought and
action of the perceiver who may respond rhetorically if he is in a
position to do so.

\textbf{Audience}\\
The second constituent is the audience. Since rhetorical discourse
produces change by influencing the decision and action of persons who
function as mediators of change, it follows that rhetoric always
requires an audience - even in those cases when a person engages himself
or ideal mind as audience. It is clear also that a rhetorical audience
must be distinguished from a body of mere hearers or readers: properly
speaking, a rhetorical audience consists only of those persons who are
capable of being influenced by discourse and of being mediators of
change.

\textbf{Constraints}\\
Besides exigence and audience, every rhetorical situation contains a set
of constraints made up of persons, events, objects, and relations which
are parts of the situation because they have the power to constrain
decision and action needed to modify the exigence. Standard sources of
constraint include beliefs, attitudes, documents, facts, traditions,
images, interests, motives and the like; and when the orator enters the
situation, his discourse not only harnesses constraints given by
situation but provides additional important constraints - for example
his personal character, his logical proofs, and his style. There are two
main classes of constraints: (1) those originated or managed by the
rhetor and his method (Aristotle called these ``artistic proofs''), and
(2) those other constraints, in the situation, which may be operative
(Aristotle's ``inartistic proofs'').

\bibliography{book.bib,packages.bib}


\end{document}
